% Copyright 2021 Melvin Eloy Irizarry-Gelpí
\chapter{Slide Deck Style Guide}
%
\begin{itemize}
    \item Include your name, as well as the names of your laboratory partners.
    \item Each \textbf{chart} needs a \textbf{title}, and \textbf{axes labels}, as well as a \textbf{brief description}.
    \item Each \textbf{table} needs a \textbf{title}, \textbf{column labels}, row labels (if necessary), as well as a \textbf{brief description}.
    \item Axis/row/column \textbf{labels} should indicate the \textbf{name} and \textbf{units} of the quantity.
    \item Please, make sure that a given \textbf{table} is included entirely in a \textbf{single slide}. Avoid tables with dangling rows across multiple slides.
    \item \textbf{Descriptions} for charts/tables should explain in \textbf{words} what the chart/table is presenting, as well as what physical principle is being verified. Each description must be a few \textbf{complete sentences} long.
    \item Please, include the description for a given chart or table in the \textbf{same slide} as the chart or table.
    \item Read the slide deck document carefully and provide \textbf{answers to all the questions} that are included in it. These answers must be in the form of a \textbf{complete sentence}. Preface your answer with the original question for context.
    \item The slide deck document includes \textbf{table templates with suggested organization}. You are allowed to use your own organization, but you must include all the requested information. I strongly encourage you to use the suggested table organization.
    \item Please, do not break multiple runs into individual tables unless explicitly asked to. (That is, do not have a table for run 1, and then a separate table for run 2...)
    \item You can collaborate with your laboratory partners, but you are expected to submit only your work.
    \item Make sure to use the appropriate number of decimal figures.
\end{itemize}