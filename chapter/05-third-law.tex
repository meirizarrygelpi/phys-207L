% Copyright 2018 Melvin Eloy Irizarry-Gelpí
\setcounter{chapter}{5}
\chapter{Newton's Third Law}
%%%%%%%%%%%%%%%%%%%%%%%%%%%%%%%%%%%%%%%%%%%%%%%%%%%%%%%%%%%%%%%%%%%%%%%%%%%%%%%%
In this experiment you will check the validity of Newton's third law of motion.
%%%%%%%%%%%%%%%%%%%%%%%%%%%%%%%%%%%%%%%%%%%%%%%%%%%%%%%%%%%%%%%%%%%%%%%%%%%%%%%%
\section{Preliminary}
%%%%%%%%%%%%%%%%%%%%%%%%%%%%%%%%%%%%%%%%%%%%%%%%%%%%%%%%%%%%%%%%%%%%%%%%%%%%%%%%
The \textbf{third law of motion} can be stated as follows:
\begin{itemize}
    \item Whenever A exerts a force $\vec{F}_{AB}$ on B, B exerts a force $\vec{F}_{BA}$ on A.
    \item The \textbf{magnitude} of $\vec{F}_{BA}$ and $\vec{F}_{AB}$ are equal.
    \item The \textbf{direction} of $\vec{F}_{BA}$ is the exact opposite of the direction of $\vec{F}_{AB}$.
\end{itemize}
In more mathematical terms,
\begin{align}
    F_{BA} &= F_{AB} & \vec{F}_{BA} &= - \vec{F}_{AB}
\end{align}
One way to test the third law would be to measure the force acting on two bodies and verify that the magnitudes are the same, but that the direction are opposite.

There is a common misconception about the third law. Sometimes, it is stated as
\begin{itemize}
    \item For every action, there is an equal and opposite \textbf{reaction}.
\end{itemize}
This wording leads to an \textbf{incorrect} picture because it introduces an order in time, as in object A first acts on object B, and then, as a reaction, object B acts on object A. This is incorrect. Both forces in Newton's third law appear \textbf{at the same time}. You can check this by measuring the force across time. If the reaction picture were accurate, then you would see a time-delay in one of the forces. Another way to state this is that there is a symmetry between the objects in the sense that there is no breakdown into aggressor and receiver.
%%%%%%%%%%%%%%%%%%%%%%%%%%%%%%%%%%%%%%%%%%%%%%%%%%%%%%%%%%%%%%%%%%%%%%%%%%%%%%%%
\section{Experiment}
%%%%%%%%%%%%%%%%%%%%%%%%%%%%%%%%%%%%%%%%%%%%%%%%%%%%%%%%%%%%%%%%%%%%%%%%%%%%%%%%
You used two force sensors to measure the amount of force acting on each other. To check different aspects of the third law, you perform three kinds of experiments:
\begin{itemize}
    \item Sensor A pulls/pushes on sensor B.
    \item Sensor B pulls/pushes on sensor A.
    \item Both sensors pull/push on each other.
\end{itemize}
Furthermore, you tried different attachments on the force sensors:
\begin{itemize}
    \item String (pulling)
    \item Rubber band (pulling)
    \item Bumpers (pushing)
    \item Magnets (pushing)
\end{itemize}
The force sensors give you force values over time.
%%%%%%%%%%%%%%%%%%%%%%%%%%%%%%%%%%%%%%%%%%%%%%%%%%%%%%%%%%%%%%%%%%%%%%%%%%%%%%%%
\section{Analysis}
%%%%%%%%%%%%%%%%%%%%%%%%%%%%%%%%%%%%%%%%%%%%%%%%%%%%%%%%%%%%%%%%%%%%%%%%%%%%%%%%
...
%%%%%%%%%%%%%%%%%%%%%%%%%%%%%%%%%%%%%%%%%%%%%%%%%%%%%%%%%%%%%%%%%%%%%%%%%%%%%%%%
\section{My Data}
%%%%%%%%%%%%%%%%%%%%%%%%%%%%%%%%%%%%%%%%%%%%%%%%%%%%%%%%%%%%%%%%%%%%%%%%%%%%%%%%
...
%%%%%%%%%%%%%%%%%%%%%%%%%%%%%%%%%%%%%%%%%%%%%%%%%%%%%%%%%%%%%%%%%%%%%%%%%%%%%%%%
\section{Your Data}
%%%%%%%%%%%%%%%%%%%%%%%%%%%%%%%%%%%%%%%%%%%%%%%%%%%%%%%%%%%%%%%%%%%%%%%%%%%%%%%%
...
%%%%%%%%%%%%%%%%%%%%%%%%%%%%%%%%%%%%%%%%%%%%%%%%%%%%%%%%%%%%%%%%%%%%%%%%%%%%%%%%
\newpage
\section{Your Laboratory Report}
%%%%%%%%%%%%%%%%%%%%%%%%%%%%%%%%%%%%%%%%%%%%%%%%%%%%%%%%%%%%%%%%%%%%%%%%%%%%%%%%
...
%%%%%%%%%%%%%%%%%%%%%%%%%%%%%%%%%%%%%%%%%%%%%%%%%%%%%%%%%%%%%%%%%%%%%%%%%%%%%%%%
\newpage
\section{Tables}
%%%%%%%%%%%%%%%%%%%%%%%%%%%%%%%%%%%%%%%%%%%%%%%%%%%%%%%%%%%%%%%%%%%%%%%%%%%%%%%%
\FloatBarrier
\newpage
\section{Figures}
%%%%%%%%%%%%%%%%%%%%%%%%%%%%%%%%%%%%%%%%%%%%%%%%%%%%%%%%%%%%%%%%%%%%%%%%%%%%%%%%
