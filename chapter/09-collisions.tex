% Copyright 2018-2020 Melvin Eloy Irizarry-Gelpí
% \setcounter{chapter}{8}
\chapter{Collisions and Momentum}
%
In this experiment you will study momentum and kinetic energy in collisions.
%
\section{Preliminary}
%
Energy and linear momentum are very important physical quantities. These quantities can make certain problems easier to solve. One class of such problems are collisions.

Consider a system with two objects, labeled 1 and 2. Each object has a \textbf{mass} ($m_{1}$ and $m_{2}$). A collision is a very quick event where object 1 and 2 interact. Before the collision, each object has an \textbf{initial velocity} ($u_{1}$ and $u_{2}$). With the mass and the initial velocities, you can find the \textbf{initial momentum} of each object ($p_{1}$ and $p_{2}$) given by
\begin{align}
    p_{1} &= m_{1} u_{1} & p_{2} &= m_{2} u_{2}
\end{align}
Then you can find the \textbf{total initial linear momentum} $p$ given by the sum of the individual initial momenta:
\begin{equation}
    p = p_{1} + p_{2}
\end{equation}
You can also find the \textbf{initial kinetic energy} of each object ($K_{1}$ and $K_{2}$):
\begin{align}
    K_{1} &= \frac{1}{2} m_{1} \left(u_{1}\right)^{2} & K_{2} &= \frac{1}{2} m_{2} \left(u_{2}\right)^{2}
\end{align}
and the \textbf{total initial kinetic energy} $K$ given by the sum of the individual initial energies:
\begin{equation}
    K = K_{1} + K_{2}
\end{equation}
Immediately after the collision, each object has a \textbf{final velocity} ($v_{1}$ and $v_{2}$). With the mass and the final velocities, you can find the \textbf{final momentum} of each object ($q_{1}$ and $q_{2}$) given by
\begin{align}
    q_{1} &= m_{1} v_{1} & q_{2} &= m_{2} v_{2}
\end{align}
Then you can find the \textbf{total final linear momentum} $q$ given by the sum of the individual final momenta:
\begin{equation}
    q = q_{1} + q_{2}
\end{equation}
You can also find the \textbf{final kinetic energy} of each object ($L_{1}$ and $L_{2}$):
\begin{align}
    L_{1} &= \frac{1}{2} m_{1} \left(v_{1}\right)^{2} & L_{2} &= \frac{1}{2} m_{2} \left(v_{2}\right)^{2}
\end{align}
and the \textbf{total final kinetic energy} $L$ given by the sum of the individual final energies:
\begin{equation}
    L = L_{1} + L_{2}
\end{equation}
With some many quantities available, you might be having the following questions:
\begin{itemize}
    \item Are the values after the collision related to the values before the collision?
    \item Are the values associated to object 1 related to the values associated to object 2?
\end{itemize}
In a previous experiment you encountered a quantity whose value was approximately constant in time: the mechanic energy. When a cart moved along an inclined track, you saw that the value of both kinetic and gravitational energy changed with time, but the value of the sum (the mechanic energy) stayed almost constant. Back then you learned that
\begin{itemize}
    \item Energy cannot be created nor destroyed; only \textbf{converted} into other forms of energy.
\end{itemize}
That previous experiment involved a single object, the cart. A collision involves more than one object. As you are going to check, the above statement on energy can be modified to
\begin{itemize}
    \item Energy cannot be created nor destroyed; it can be \textbf{transferred} to another object or \textbf{converted} into other forms of energy.
\end{itemize}
In this experiment you will see that, in general, the total amount of momentum is conserved. However, the total amount of kinetic energy is not always conserved.
%
\section{Experiment}
%
You are going to consider three kinds of collisions:
\begin{enumerate}
    \item Elastic collisions
    \item Inelastic collisions
    \item Explosive collisions
\end{enumerate}
Each of these leads to different results.
%
\subsection{Elastic Collisions}
%
In the \textbf{elastic} collision, the objects bounce off each other after colliding. To create this effect you are going to use \textbf{repelling magnets}. The particular elastic collision you are going to study is as follows:
\begin{itemize}
    \item Before the collision, object 1 is moving towards object 2, but object 2 is at rest on the track.
    \item After the collision, object 1 might move (either backwards or forwards), and object 2 will also move.
\end{itemize}
You can change the amount of mass on each cart to see different versions.
%
\subsection{Inelastic Collisions}
%
In the \textbf{inelastic} collision, the objects stick together after colliding. To create this effect you are going to use \textbf{velcro}. The particular inelastic collision you are going to study is as follows:
\begin{itemize}
    \item Before the collision, object 1 is moving towards object 2, but object 2 is at rest on the track.
    \item After the collision, both object 1 and object 2 move together.
\end{itemize}
You can change the amount of mass on each cart to see different versions.
%
\subsection{Explosive Collisions}
%
For the \textbf{explosive} collision you are going to use the \textbf{plunger} on one of the carts. This ``collision'' is as follows:
\begin{itemize}
    \item Before the ``collision'', both objects sit at rest on the track.
    \item After the ``collision'', both objects move away from each other.
\end{itemize}
Strictly speaking, this is not a collision in the usual sense. However, if you record a video of the event and play it backwards, you would see both carts moving towards each other, colliding, and becoming at rest.

Again, you can change the amount of mass on each cart to see different versions.
%
\section{Analysis}
%
The idea is to measure the velocities of each cart and use those values to calculate the quantities mentioned above. The most difficult part is to figure out which velocity value correspond to which cart.

As a way to compare the total momentum and total kinetic energy before and after the collision, you can calculate the ratio:
\begin{align}
    \frac{q}{p} && \frac{L}{K}
\end{align}
The structure of these ratio is ``value after divided by value before collision''. If the ratio is less than one, then the value before the collision is larger, otherwise the value after the collision is larger. Note that these ratios cannot be calculated for explosive collisions.
%
\section{My Data}
%
I have three data sets:
\begin{itemize}
    \item One data set for elastic collisions.
    \item One data set for inelastic collisions.
    \item One data set for explosive collisions.
\end{itemize}
Each of these data sets has 9 runs of data:
\begin{itemize}
    \item Runs 1, 2, and 3: both carts have no extra mass.
    \item Runs 4, 5, and 6: cart 2 has 500 g of extra mass.
    \item Runs 7, 8, and 9: cart 1 has 500 g of extra mass.
\end{itemize}
There are three runs per mass variation to check for consistency. The bare total mass of each cart can be found in Table \ref{09:table.mass}.
%
\section{Your Data}
%
Your data will have the exact same structure as my data.
%
\newpage
\section{Your Laboratory Report}
%
Your lab report should include the following:
\begin{itemize}
    \item A table like Table \ref{09:table.mass} with the bare total mass of each cart (mass of cart plus mass of picket fence).
    \item Tables like \ref{09:table.v.elastic}, \ref{09:table.v.inelastic}, and \ref{09:table.v.explosive} with the initial and final velocities measured, along with the particular mass values used.
    \item Tables like \ref{09:table.p.elastic}, \ref{09:table.p.inelastic}, and \ref{09:table.p.explosive} with the initial and final momenta calculated. Also include the total initial momentum and total final momentum, as well as the ratio of the total final momentum to the total initial momentum.
    \item Tables like \ref{09:table.K.elastic}, \ref{09:table.K.inelastic}, and \ref{09:table.K.explosive} with the initial and final kinetic energies calculated. Also include the total initial kinetic energy and total final kinetic energy, as well as the ratio of the total final kinetic energy to the total initial kinetic energy.
\end{itemize}
You should also answer the following questions:
\begin{enumerate}
    \item In general, is the total momentum before the collision ($p$) comparable to the total momentum after the collision ($q$)? When does this statement does not hold? Why?
    \item In general, is the total kinetic energy before the collision ($K$) comparable to the total kinetic energy after the collision ($L$)? When does this statement does not hold? Why?
\end{enumerate}
%
\newpage
\section{Example Tables}
%
\begin{table}[ht]
    \centering
    \begin{tabular}{l|r}
        \textbf{Name} & \textbf{Value} (kg) \\
        \hline
        Bare Total Mass of Cart 1 & 0.280 \\
        Bare Total Mass of Cart 2 & 0.284 \\
        \hline
    \end{tabular}
    \caption{Bare total mass values for each cart.}
    \label{09:table.mass}
\end{table}
%
\begin{table}[ht]
    \centering
    \begin{tabular}{l|r|r|r|r|r|r}
        \textbf{Run} & $m_{1}$ (kg) & $m_{2}$ (kg) & $u_{1}$ (m/s) & $u_{2}$ (m/s) & $v_{1}$ (m/s) & $v_{2}$ (m/s) \\
        \hline
        1 & & & & & & \\
        2 & & & & & & \\
        3 & & & & & & \\
        \hline
        4 & & & & & & \\
        5 & & & & & & \\
        6 & & & & & & \\
        \hline
        7 & & & & & & \\
        8 & & & & & & \\
        9 & & & & & & \\
        \hline
    \end{tabular}
    \caption{Initial and final velocities for elastic collisions}
    \label{09:table.v.elastic}
\end{table}
%
\begin{table}[ht]
    \centering
    \begin{tabular}{l|r|r|r|r|r|r}
        \textbf{Run} & $m_{1}$ (kg) & $m_{2}$ (kg) & $u_{1}$ (m/s) & $u_{2}$ (m/s) & $v_{1}$ (m/s) & $v_{2}$ (m/s) \\
        \hline
        1 & & & & & & \\
        2 & & & & & & \\
        3 & & & & & & \\
        \hline
        4 & & & & & & \\
        5 & & & & & & \\
        6 & & & & & & \\
        \hline
        7 & & & & & & \\
        8 & & & & & & \\
        9 & & & & & & \\
        \hline
    \end{tabular}
    \caption{Initial and final velocities for inelastic collisions}
    \label{09:table.v.inelastic}
\end{table}
%
\begin{table}[ht]
    \centering
    \begin{tabular}{l|r|r|r|r|r|r}
        \textbf{Run} & $m_{1}$ (kg) & $m_{2}$ (kg) & $u_{1}$ (m/s) & $u_{2}$ (m/s) & $v_{1}$ (m/s) & $v_{2}$ (m/s) \\
        \hline
        1 & & & & & & \\
        2 & & & & & & \\
        3 & & & & & & \\
        \hline
        4 & & & & & & \\
        5 & & & & & & \\
        6 & & & & & & \\
        \hline
        7 & & & & & & \\
        8 & & & & & & \\
        9 & & & & & & \\
        \hline
    \end{tabular}
    \caption{Initial and final velocities for explosive collisions}
    \label{09:table.v.explosive}
\end{table}
%
\begin{table}[ht]
    \centering
    \begin{tabular}{l|r|r|r|r|r|r|r}
        \textbf{Run} & $p_{1}$ & $p_{2}$ & $p = p_{1} + p_{2}$ & $q_{1}$ & $q_{2}$ & $q = q_{1} + q_{2}$ & $q / p$ \\
        \hline
        1 & & & & & & & \\
        2 & & & & & & & \\
        3 & & & & & & & \\
        \hline
        4 & & & & & & & \\
        5 & & & & & & & \\
        6 & & & & & & & \\
        \hline
        7 & & & & & & & \\
        8 & & & & & & & \\
        9 & & & & & & & \\
        \hline
    \end{tabular}
    \caption{Momentum for elastic collisions. All quantities are in units of kg m/s.}
    \label{09:table.p.elastic}
\end{table}
%
\begin{table}[ht]
    \centering
    \begin{tabular}{l|r|r|r|r|r|r|r}
        \textbf{Run} & $p_{1}$ & $p_{2}$ & $p = p_{1} + p_{2}$ & $q_{1}$ & $q_{2}$ & $q = q_{1} + q_{2}$ & $q / p$ \\
        \hline
        1 & & & & & & & \\
        2 & & & & & & & \\
        3 & & & & & & & \\
        \hline
        4 & & & & & & & \\
        5 & & & & & & & \\
        6 & & & & & & & \\
        \hline
        7 & & & & & & & \\
        8 & & & & & & & \\
        9 & & & & & & & \\
        \hline
    \end{tabular}
    \caption{Momentum for inelastic collisions. All quantities are in units of kg m/s.}
    \label{09:table.p.inelastic}
\end{table}
%
\begin{table}[ht]
    \centering
    \begin{tabular}{l|r|r|r|r|r|r}
        \textbf{Run} & $p_{1}$ & $p_{2}$ & $p = p_{1} + p_{2}$ & $q_{1}$ & $q_{2}$ & $q = q_{1} + q_{2}$ \\
        \hline
        1 & & & & & & \\
        2 & & & & & & \\
        3 & & & & & & \\
        \hline
        4 & & & & & & \\
        5 & & & & & & \\
        6 & & & & & & \\
        \hline
        7 & & & & & & \\
        8 & & & & & & \\
        9 & & & & & & \\
        \hline
    \end{tabular}
    \caption{Momentum for explosive collisions. All quantities are in units of kg m/s.}
    \label{09:table.p.explosive}
\end{table}
%
\begin{table}[ht]
    \centering
    \begin{tabular}{l|r|r|r|r|r|r|r}
        \textbf{Run} & $K_{1}$ & $K_{2}$ & $K = K_{1} + K_{2}$ & $L_{1}$ & $L_{2}$ & $L = L_{1} + L_{2}$ & $L / K$ \\
        \hline
        1 & & & & & & & \\
        2 & & & & & & & \\
        3 & & & & & & & \\
        \hline
        4 & & & & & & & \\
        5 & & & & & & & \\
        6 & & & & & & & \\
        \hline
        7 & & & & & & & \\
        8 & & & & & & & \\
        9 & & & & & & & \\
        \hline
    \end{tabular}
    \caption{Kinetic energy for elastic collisions. All quantities are in units of J.}
    \label{09:table.K.elastic}
\end{table}
%
\begin{table}[ht]
    \centering
    \begin{tabular}{l|r|r|r|r|r|r|r}
        \textbf{Run} & $K_{1}$ & $K_{2}$ & $K = K_{1} + K_{2}$ & $L_{1}$ & $L_{2}$ & $L = L_{1} + L_{2}$ & $L / K$ \\
        \hline
        1 & & & & & & & \\
        2 & & & & & & & \\
        3 & & & & & & & \\
        \hline
        4 & & & & & & & \\
        5 & & & & & & & \\
        6 & & & & & & & \\
        \hline
        7 & & & & & & & \\
        8 & & & & & & & \\
        9 & & & & & & & \\
        \hline
    \end{tabular}
    \caption{Kinetic energy for inelastic collisions. All quantities are in units of J.}
    \label{09:table.K.inelastic}
\end{table}
%
\begin{table}[ht]
    \centering
    \begin{tabular}{l|r|r|r|r|r|r}
        \textbf{Run} & $K_{1}$ & $K_{2}$ & $K = K_{1} + K_{2}$ & $L_{1}$ & $L_{2}$ & $L = L_{1} + L_{2}$ \\
        \hline
        1 & & & & & & \\
        2 & & & & & & \\
        3 & & & & & & \\
        \hline
        4 & & & & & & \\
        5 & & & & & & \\
        6 & & & & & & \\
        \hline
        7 & & & & & & \\
        8 & & & & & & \\
        9 & & & & & & \\
        \hline
    \end{tabular}
    \caption{Kinetic energy for explosive collisions. All quantities are in units of J.}
    \label{09:table.K.explosive}
\end{table}
%